\documentclass[a4paper,12pt]{article}
% Additional Options
\usepackage{booktabs}
\usepackage{enumerate}
\usepackage{multirow}
\usepackage{subfigure}
\usepackage{color}
\usepackage{graphicx}
\usepackage{amsmath}
\usepackage[section]{placeins}
\usepackage{geometry}
\usepackage{float}
\usepackage{xcolor}
\usepackage{listings}
\geometry{left=3.0cm, right=3.0cm, top=2.5cm, bottom=2.5cm}
\usepackage{cite}
\usepackage{setspace}%
\usepackage{indentfirst}
\renewcommand\thesection{\Roman{section}}

\setlength{\parindent}{1.5em}
%\setlength\parskip{.3\baselineskip}

\lstset{ %  
	backgroundcolor=\color{white},   % choose the background color; you must add \usepackage{color} or \usepackage{xcolor}  
	basicstyle=\footnotesize,        % the size of the fonts that are used for the code  
	breakatwhitespace=false,         % sets if automatic breaks should only happen at whitespace  
	breaklines=true,                 % sets automatic line breaking  
	captionpos=bl,                    % sets the caption-position to bottom  
	commentstyle=\color{mygreen},    % comment style  
	deletekeywords={...},            % if you want to delete keywords from the given language  
	escapeinside={\%*}{*)},          % if you want to add LaTeX within your code  
	extendedchars=true,              % lets you use non-ASCII characters; for 8-bits encodings only, does not work with UTF-8  
	frame=single,                    % adds a frame around the code  
	keepspaces=true,                 % keeps spaces in text, useful for keeping indentation of code (possibly needs columns=flexible)  
	keywordstyle=\color{blue},       % keyword style  
	%language=Python,                 % the language of the code  
	morekeywords={*,...},            % if you want to add more keywords to the set  
	numbers=left,                    % where to put the line-numbers; possible values are (none, left, right)  
	numbersep=5pt,                   % how far the line-numbers are from the code  
	numberstyle=\tiny\color{gray}, % the style that is used for the line-numbers  
	rulecolor=\color{black},         % if not set, the frame-color may be changed on line-breaks within not-black text (e.g. comments (green here))  
	showspaces=false,                % show spaces everywhere adding particular underscores; it overrides 'showstringspaces'  
	showstringspaces=false,          % underline spaces within strings only  
	showtabs=false,                  % show tabs within strings adding particular underscores  
	stepnumber=1,                    % the step between two line-numbers. If it's 1, each line will be numbered  
	stringstyle=\color{orange},     % string literal style  
	tabsize=2,                       % sets default tabsize to 2 spaces  
	%title=myPython.py                   % show the filename of files included with \lstinputlisting; also try caption instead of title  
}  


\begin{document}

\begin{spacing}{1.3}%
\vspace*{0.25cm}

\hrulefill

\thispagestyle{empty}

\begin{center}
\begin{large}
\sc{New York University \vspace{0.3em} \\
	Tandon School of Engineering \vspace{0.3em} \\ CS 6513: Big Data}
\end{large}

\hrulefill

\vspace*{2cm}

\begin{large}
\sc{{Project Final Report}}
\end{large}
\vspace*{2cm}

\begin{large}
	\sc{{The Effect of COVID-19 Government Policies on Social Distancing and Infection Numbers}}
\end{large}

\end{center}

\vfill

\begin{table}[h!]
\begin{center}
\begin{tabular}{ll}
	Group Name:  OneTeammateRanAway\hspace*{2em} & \\
	&\\
	
	Name: Haifeng Zhang \hspace*{2em}&
	netID: hz2149\\
	
Name: Tianyi Zhao \hspace*{2em}&
netID: tz1330\\

 \hspace*{2em} \\
 \\

Date: May 10 2020


\end{tabular}
\end{center}
\end{table}

\newpage


\section{Introduction}

In the COVID-19 pandemic, social distancing has been a major mechanism to prevent outbreaks and to reduce the risk of infection. U.S state governments have raised policies on social contacts such as school closure, restaurant limits and other stay-at-home policies. However, the public may take measures not immediately or before the governments publish orders. This project will study on the community mobility change to the safety policies.

The project will focus on the effects of governments' stay-home policies on social distancing in the United States, especially in the New York State. We will use the public mobility data by Google and an open source government policy data. Furthermore, we will combine the infection numbers to our results and evaluate how effective the policies are in reducing COVID-19 spread.


\section{Objectives and Overview}

This project has two main problems:

\begin{itemize}
	\item How community mobility changes in response to the government stay-at-home orders in the United States.
	\item How effective is social distancing in reducing COVID-19 spread.
\end{itemize}

For the first problem, we will use community mobility data by Google to find the mobility percentage changes per day in March and April of each state in the US. And we can extract policy data of each state governments so we can match the time point to the mobility curve. We will compare the curve before and after the publish time of policies. Also, we may compare it to the same time period in 2019 or the previous weeks.

The results will be extended to the COVID-19 infection data. We will first apply a simple exponential model to the state infection curve to figure out whether the government measures reduce the increase rate. There might not be a direct conclusion since the infection number can be influenced by multiple factors. We will try to use some prediction models raised by researchers to compare the patterns of infection curve before and after governments' stay-at-home orders.



\section{Data}

The project will use 3 open source data sets:
\begin{table}
	
	\newcommand{\tabincell}[2]{\begin{tabular}{@{}#1@{}}#2\end{tabular}}
\begin{tabular}{|c|c|c|}

	\hline 
	Data set name & Link & Description\\
	\hline
\tabincell{c}{Johns Hopkins \\ dashboard data} &\tabincell{c}{https://github.com/\\CSSEGISandData/COVID-\\19/tree/master/csse\\\_covid\_19\_data}&  \tabincell{c}{Daily reports for new cases \\by Johns Hopkins CSSE, \\including daily case increase \\in each county in the United States} \\
\hline
\tabincell{c}{COVID-19 Community \\Mobility Reports} & \tabincell{c}{https://www.gstatic.com/\\covid19/mobility/Global\\\_Mobility\_Report.csv}& \tabincell{c}{
	These Community Mobility Reports \\aim to provide insights into what \\has changed in response to policies \\aimed at combating COVID-19. \\The reports chart movement trends \\over time by geography, across \\different categories of places such as \\retail and recreation, groceries \\and pharmacies, parks, transit \\stations, workplaces, and residential.} \\
\hline
\tabincell{c}{Government imposed \\safety measures} & \tabincell{c}{https://data.humdata.\\org/dataset/acaps-covid\\19-government-measures-\\dataset} & \tabincell{c}{4000-row dataset that compiles \\official public safety measures such \\as social distancing, school \\closures, etc. imposed by countries \\around the world. \\Includes the dates when these \\measures were imposed and \\the official or news source. \\Compiled by Assessment \\Capacities Project (ACAPS).}\\
\hline 
\end{tabular}

\end{table}


\section{Data Cleaning \& Integration}
The details are included in datasets-used.csv file in our project part2 github repository.


\subsection{Mobility Data}
This data set is nicely organized so we did not need to do much cleaning. We extracted the mobility percentage change since March 1st in New York.

\subsection{Infection Data}
We used time\_series\_covid19\_confirmed\_US.csv in from Johns Hopkins dashboard data. This data set is also well organized and we extracted the confirmed numbers in each counties as well as the total number in New York State.

\subsection{Government Policy Data}
We extracted information about New York government policies from the government safety measures data set. We used keywords related to New York and the United States. We extracted data related to the United States to one file and the data related only to New York to another file. This data set is much more challenging because the data are all keywords and descriptions.


\end{spacing}
\end{document}